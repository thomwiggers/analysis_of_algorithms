\documentclass{article}

\usepackage{amsmath}
\usepackage{amssymb}
\usepackage{program}
\newtheorem{algorithm}{Algorithm}[section]

\NumberProgramstrue

\author{Thom Wiggers}
\title{Analysis of Algorithms\\{\large second assignment} }
\begin{document}
\maketitle

\section{Exercise 1}

\emph{Find an algorithm in $O(n \lg n)$ which finds the x-coordinate of
a line which intersects the maximum amount of line segments.}

This is a slightly modified version of the ``Any-Segment-Intersect''
algorithm of ``Introduction To Algoritms'', page 1025. It keeps for 
each point a score of how many lines are intersected.
\begin{algorithm}
\begin{program}
  \mbox{Find the place with the most intersections}
  \label{one}
  \BEGIN \\
    \mbox{Sort the endpoints of the segments in $S$ from left to right,}
    \quad\mbox{breaking ties by putting left endpoints before right%
      endpoints and}
    \quad\mbox{breaking further ties by putting points with lower %
      $y$-coordinates first.}
    |score| = 0 \rcomment{we first intersect nothing, obviously} \label{ass1}
    |best| = 0 \rcomment{Best score so far}  \label{ass2}
    |point| = (-\infty, -\infty)  \rcomment{Best point so far, (x,y)} \label{ass3}
    \FOREACH {point $p$ in the sorted list of endpoints} \label{forb}
      \IF $p$ \mbox{ is the left endpoint of a segment } s \label{if1}
        \THEN 
          |inc|(score) \rcomment{increment score}
          \IF |score| > |best| \label{if2}
            \THEN 
              |point| = p
              |best| = |score|
          \FI
        \ELSE
          |dec|(|score|) \rcomment{We're leaving a line}
      \FI
    \END \label{forend}
    |return | |point.x|
  \END
\end{program}
\end{algorithm}

As I've shown, no data structure is needed.

\subsection{Complexity}
This program has complexity $O(n \lg n)$.

The initial sorting has complexity $O(n \lg n)$. Then the algorithm has
complexity $n$:

If set $S$ contains $n$ segments, then lines \ref{ass1} - \ref{ass3} run in
$O(1)$. The foreach loop of lines \ref{forb} - \ref{forend} iterates at most
once per point, and with $2n$ points it iterates $2n$ times. All instructions
in the for loop only take $O(1)$ time, since they are only comparisons and
assignments. So the second part of the algoritm only takes $O(n)$ time.

\subsection{Correctness}

I choose the following loop invariant: 'point' always contains a point with
the most intersections left of the sweep line (1), 'best' contains the amount
of intersections at 'point' (2) and 'score' always contains the amount of
intersections at the current $p$ (3).

This is true prior to iteration because there are no points left to the sweep
line, and thus the 'best' and current amount of intersections is '0' (line
\ref{ass1} and \ref{ass2}), and the point is minus $\infty$ left of the sweep
line. 

In the loop there are two main cases: either $p$ is a left endpoint or $p$ is
a right endpoint.

Assume that $p$ is a left endpoint. Then we've arrived at a new line which our
sweep line intersects. We increments 'score', which contained the amount of
intersections (3), so it contains the new amount of intersections. (3) thus
still holds after this iteration, because it is not changed again. Now, if
'score' is greater than 'best', $p$ is better than 'point', because 'best' 
contained the amount of intersections at 'point'(2). So 'point' and
'best' are updated, again satisfying (1) and (2). If 'score' isn't greater than
'best', 'best' and 'point' still are correct.

Assume that $p$ is a right endpoint. Then we've left a line. To satisfy (3)
again we decrement 'score'. Because we're leaving a point, 'best' and 'point'
still satisfy (1) and (2).

So in every case we satisfy the loop invariant. After completing the loop, 
all points are to the left of the sweep line, and 'point' has a point where
the most intersections happen.

\section{Exercise 2}

CH(P) already is sorted, sorting $\{p\}$ into that list can be done in $O(n)$
time. You then only need to execute lines 3 -- 11 of the Graham-Scan algoritm.
The only question that needs to be answered is if $p$ is an outer point or not,
therefore there is no need to look at all the points of $P$ again: it can only
be an outer point if it's outside the current outer points which are in CH(P).
Lines 3--11 are $O(n)$ (page 1036, introduction to algoritms).


\end{document}
